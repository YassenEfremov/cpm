\graphicspath{ {./chapter4/images/} }

\chapter{Ръководство за потребителя}


\section{Инсталиране на програмата}

Програмата е достъпна както за Linux, така и за Windows. Преди да започнете
процеса на инсталация ще са Ви нужни следните допълнителни програми:

\begin{itemize}
    \item git
    \item CMake (по желание заедно с ninja\footnote{Може да използвате build
          системата ninja за по-бърза компилация})
	\item C++ компилатор (g++ или MSVC)
\end{itemize}


\subsection{Сдобиване с кода на програмата}

Кодът на програмата може да изтеглите от GitHub хранилището на проекта:
\url{https://github.com/YassenEfremov/cpm}. Най-удобният и бърз начин е да
клонирате хранилището директно:

\begin{lstlisting}[style=shell]
$ git clone https://github.com/YassenEfremov/cpm
\end{lstlisting}

Като алтернатива може да изтеглите кода под формата на zip архив от страницата с
официални публикации \figref{fig:download-zip}:

\begin{figure}[h]
    \centering
    \includegraphics[width=.9\textwidth]{release}
    \caption{Теглене на кода под формата на zip архив}
    \label{fig:download-zip}
\end{figure}


\subsection{Конфигуриране и компилиране на програмата}

След като сте се сдобили с кода може да преминете към компилиране на
програмата. Стъпките за това са следните:

\begin{enumerate}
    \item Инициализиране на git submodules - проектът използва git \\
          submodules за управление на зависимостите. Преди да можете да
          компилирате основната програма, трябва да изтеглите всички нейни
          зависимости. Това става лесно със следната команда:
    
\begin{lstlisting}[style=shell]
$ git submodule update --init --progress
\end{lstlisting}

    \item Конфигуриране на проекта - за да можете да компилирате проекта трябва
          първо да генерирате build файловете, необходими за вашата конкретна
          операционна система. За целта е препоръчително да създадете нова
          директория, в която ще бъдат генерирани съответните build файлове:

\begin{lstlisting}[style=shell]
$ mkdir build
$ cd build
\end{lstlisting}

След като сте в новосъздадената директория можете да генерирате съответните
build файлове като изпълните командата \texttt{cmake} и подадете директорията,
съдържаща основния \texttt{CMakeLists.txt} файл на проекта, в този случай това
е предишната директория, обозначена с две точки:

\begin{lstlisting}[style=shell]
$ cmake ..
\end{lstlisting}

Това автоматично ще избере подходящата build система в зависимост от текущата
операционна система. Ако желаете да използвате конкретна build система можете да
подадете на \texttt{cmake} параметъра \texttt{-G} и името на build системата. За
текущия проект е препоръчително да използвате build системата ninja за по-бърза
компилация:

\begin{lstlisting}[style=shell]
$ cmake .. -G Ninja
\end{lstlisting}

    \item Компилиране на програмата - компилирането на програмата става като се
          изпълнят build файлове, генерирани в предишната стъпка. Следната
          команда автоматично разпознава генерираните build файлове и ги
          изпълнява:

\begin{lstlisting}[style=shell]
$ cmake --build
\end{lstlisting}

    \item Инсталиране на програмата - за да можете да достъпвате програмата от
          която и да е папака на компютъра трябва да я инсталирате. Това изисква
          администраторски права и става чрез следната команда:

\begin{lstlisting}[style=shell]
$ sudo cmake --install
\end{lstlisting}

\end{enumerate}


\section{Синтаксис}

След като сте инсталирали програмата, можете да я пуснете чрез изпълнимия файл
\texttt{cpm}. Ако не подадете никакви параметри се извежда обобщено описание на
програмата \figref{fig:cpm}:

\begin{figure}[h]
    \centering
    \includegraphics[width=1\textwidth]{cpm}
    \caption{Обобщено описание на програмата}
    \label{fig:cpm}
\end{figure}

По-горното описание показва начина на използване на програмата и параметрите,
които приема. Това съобщение може да бъде показано и като подадете параметъра
\texttt{-{}-help} (или по-краткото \texttt{-h}). Може да видите коя версия на
програмата е инсталирана като подадете параметъра \texttt{-{}-version} (или
по-краткото \texttt{-v}) \figref{fig:version}:

\begin{figure}[H]
    \centering
    \includegraphics[width=1\textwidth]{version}
    \caption{Показване на версията}
    \label{fig:version}
\end{figure}


\subsection{Инсталиране на пакети}

Инсталирането на пакети става чрез командата \texttt{install}. Може да видите
нейното обобщено описание като подадете параметъра \\
\texttt{-{}-help} (или по-краткото \texttt{-h}) \figref{fig:install-help}:

\begin{figure}[h]
    \centering
    \includegraphics[width=1\textwidth]{install-help}
    \caption{Обобщено описание на командата за инсталиране}
    \label{fig:install-help}
\end{figure}

Командата приема произволен брой параметри, всеки от които идентифицира даден
пакет, който трябва да бъде инсталиран. При подаване на пакети се използва
синтаксиса \texttt{<име на пакет>@<версия>}. Ако версията не е подадена
програмата автоматично намира последната такава. Пакетите могат да бъдат
инсталирани по два начина:

\begin{itemize}
    \item локално (по подразбиране) - пакетите се инсталират в папката
          \texttt{<cwd>/lib/} (където \texttt{<cwd>} е текущата работна
          директория) и се добавят като зависимости на текущия пакет.
          Този начин на иснталация е предназначен за библиотеки, които
          единствено текущият пакет ще използва.
    \item глобално - пакетите се инсталират в глобалната директория на
          програмата и не се добавят като зависимости на текущия пакет. Този
          начин на иснталация е предназначен за библиотеки, които ще бъдат
          използвани от повече от един пакет. Може да бъде избран чрез подаване
          на флага \texttt{-{}-global} (или по-краткото \texttt{-g}).
\end{itemize}

Ето как изглежда инсталацията на примерния пакет \texttt{histogram} и неговата
зависимост - пакетът \texttt{ppm} \figref{fig:install-progress}:

\begin{figure}[H]
    \centering
    \includegraphics[width=1\textwidth]{install-progress}
    \caption{Инсталиране на пакет и неговите зависимости}
    \label{fig:install-progress}
\end{figure}

По време на инсталиране на пакетите се показват анимирани индикатори за
напредък. Всеки пакет има собствен индикатор, състоящ се от заглавие, определящо
извършващото се действие, и лента, опресняваща се всеки път, когато има
напредък. Синият символ ``v'' пред индикатора означава теглене на данни.
Зеленият символ ``+'' пред индикатора означава декомпресиране на данни.

След като пакетът е инсталиран, програмата предоставя на потребителя инструкции
за това как да го използва чрез build системата CMake \figref{fig:install}.
Инструкциите са оцветени в жълто, за да бъдат лесно различими от останалите
съобщения.

\begin{figure}[h]
    \centering
    \includegraphics[width=1\textwidth]{install}
    \caption{Инструкции за използване на инсталираните пакети}
    \label{fig:install}
\end{figure}

Ако някой от подадените пакети е вече инсталиран, командата сигнализира на
потребителя като извежда грешка и отново покзва инструкции относно начина на
използване на пакета \figref{fig:installed}:

\begin{figure}[H]
    \centering
    \includegraphics[width=1\textwidth]{installed}
    \caption{Опит за инсталиране на вече инсталиран пакет}
    \label{fig:installed}
\end{figure}

За момента програмата не позволява инсталирането на повече от една версия на
един и същ пакет.


\subsection{Премахване на пакети}

Премахването на пакети става чрез командата \texttt{remove}. Може да видите
нейното обобщено описание като подадете параметъра \\
\texttt{-{}-help} (или по-краткото \texttt{-h}) \figref{fig:remove-help}:

\begin{figure}[h]
    \centering
    \includegraphics[width=1\textwidth]{remove-help}
    \caption{Обобщено описание на командата за премахване}
    \label{fig:remove-help}
\end{figure}

Също като командата за инсталиране, командата за премахване приема произволен
брой параметри, всеки от които идентифицира даден пакет, който трябва да бъде
премахнат. Отново пакетите могат да бъдат премахнати по два начина - локално
(по подразбиране) и глобално, като параметрите към програмата са аналогични
(\texttt{-{}-global} или \texttt{-g}).

Ето как изглежда премахването на примерния пакет \texttt{histogram} и неговите
зависимости \figref{fig:remove}:

\begin{figure}[h]
    \centering
    \includegraphics[width=1\textwidth]{remove}
    \caption{Премахване на пакет и неговите зависимости}
    \label{fig:remove}
\end{figure}

Командата за премахване не показва анимиран индикатор за напредък, а единствено
броя на премахнатите файлове. Червеният символ ``-'' пред индикатора сигнализира
изтриване на данни.

Ако някой от подадените пакети не е инсталиран, командата сигнализира на
потребителя като извежда грешка \figref{fig:not-installed}:

\begin{figure}[h]
    \centering
    \includegraphics[width=1\textwidth]{not-installed}
    \caption{Опит за премахване на пакет, който не е инсталиран}
    \label{fig:not-installed}
\end{figure}


\subsection{Изброяване на пакети}

Изброяването на пакети става чрез командата \texttt{list}. Може да видите
нейното обобщено описание като подадете параметъра \\
\texttt{-{}-help} (или по-краткото \texttt{-h}) \figref{fig:list-help}:

\begin{figure}[H]
    \centering
    \includegraphics[width=1\textwidth]{list-help}
    \caption{Обобщено описание на командата за изброяване}
    \label{fig:list-help}
\end{figure}

Командата за изброяване не приема пакети като параметри. Тя извежда имената и
версиите на инсталираните пакети. Също като предишните две команди и тази работи
в два режима - локален (по подразбиране) и глобален - като съответно в първия
случай изброява пакетите инсталирани в текущата директория, а във втория случай
тези, инсталирани глобално в глобалната директория на програмата. Параметрите
към програмата са аналогични (\texttt{-{}-global} или \texttt{-g}).

По подразбиране командата изброява единствено директно инсталираните пакети.
Ако искате да видите всички инсталирани пакети и техните зависимости трябва да
подадете параметъра \texttt{-{}-all} (или по-краткото \texttt{-a})
\figref{fig:list}. Пакетите се изреждат един под друг, като всяка зависимост е
подравнена по-навътре от предишната.

\begin{figure}[h]
    \centering
    \includegraphics[width=1\textwidth]{list}
    \caption{Изброяване на всички инсталирани пакети и техните зависимости}
    \label{fig:list}
\end{figure}


\subsection{Създаване на пакети}

Създаването на нов пакет става чрез командата \texttt{create}. Може да видите
нейното обобщено описание като подадете параметъра \\
\texttt{-{}-help} (или по-краткото \texttt{-h}) \figref{fig:create-help}:

\begin{figure}[h]
    \centering
    \includegraphics[width=1\textwidth]{create-help}
    \caption{Обобщено описание на командата за създаване}
    \label{fig:create-help}
\end{figure}

Тази команда не приема допълнителни параметри и работи само в локален режим.
При изпълнението ѝ се създава нов \texttt{cpm\_pack.json} в текущата директория
с генерирани стойности по подразбиране \figref{lst:default-cpm-pack-json}:


\begin{lstlisting}[style=json,
                   caption=cpm\_pack.json по подразбиране,
                   label={lst:default-cpm-pack-json}]
{
    "name": "example-package",
    "version": "0.1.0",
    "url": "",
    "description": "",
    "author": "",
    "license": ""
}
\end{lstlisting}


\subsection{Синхронизиране на пакети}

Зависимостите на текущия пакет могат да бъдат променени и ръчно като директно се
редактира файлът \texttt{cpm\_pack.json}. За да влязат в действие тези промени
обаче трябва да бъде изпълнена командата \texttt{sync}. Може да видите нейното
обобщено описание като подадете параметъра \texttt{-{}-help} (или по-краткото
\texttt{-h}) \figref{fig:sync-help}:

\begin{figure}[h]
    \centering
    \includegraphics[width=1\textwidth]{sync-help}
    \caption{Обобщено описание на командата за синхронизиране}
    \label{fig:sync-help}
\end{figure}

Тази команда също не приема никакви допълнителни параметри и отново работи само
в локален режим. При нейното изпълнение се инсталират всички липсващи пакети,
изброени във файла \texttt{cpm\_pack.json}, и се премахват всички излишни
пакети, които не присъстват във файла \texttt{cpm\_pack.json}. Командата
използва същите индикатори за напредък, които използват и командите за
инсталиране и премахване на пакети \figref{fig:sync}:

\begin{figure}[H]
    \centering
    \includegraphics[width=1\textwidth]{sync}
    \caption{Синхронизиране на пакети}
    \label{fig:sync}
\end{figure}

% \subsection{Глобални команди}

% По подразбиране всички команди работят в директорията, в която биват изпълнени,
% тоест командите за инсталиране, премахване и изброяване боравят с пакетите,
% намиращи се в текущата работна директория. Това е така нареченият локален режим
% на работа. Програмата предоставя още един режим - глобален. При него командите
% се изпълняват спрямо специална директория, намираща се под основната папка на
% потребителя. Разположението на специалната директория се определя от
% спецификацията за основни директории (XDG Base Directory Specification
% \cite{xdg}) и зависи от конкретната операционна система. Възможните разположения
% са следните:

% \begin{itemize}
% 	\item За Windows - \texttt{\%APPDATA\%\textbackslash cpm} (по подразбиране)
% 	\item За Linux - \texttt{\$\{HOME\}/.local/share/cpm} (по подразбиране)
% \end{itemize}

% От всички команди единствено \texttt{install}, \texttt{remove} и \texttt{list}
% могат да бъдат изпълнени в глобален режим. Това става чрез подаване на
% параметъра \texttt{-{}-global} (или по-краткото \texttt{-g}). Пример:

% \begin{lstlisting}[style=shell,
%                    caption=Изброяване на глобално инсталираните пакети]
% $ cpm list --global
% ...
% \end{lstlisting}


\section{Други функционалности}

В глобалната директория се съхраняват и някои други полезни файлове:

\begin{itemize}
    \item \texttt{log/cpm.log} - Файл, в който се записва диагностична
          информация относно действията, извършени от програмата. Този файл може
          да бъде от полза при отстраняване на грешки, възникнали по време на
          изпълнението на програмата.
    \item \texttt{package\_locations.json} - Файл, в който са изброени имената
          на GitHub профилите/организациите, от които програмата се опитва да
          тегли пакети. Файлът съдържа \acrshort{json} масива \\
          \texttt{package\_locations}, в който по подразбиране е посочена
          примерната организация \texttt{cpm-examples}
          \figref{lst:default-package-locations}. Тя може да бъде премахната
          или могат да бъдат добавени колкото и да е и каквито и да е други
          профили/организации като единственото ограничение е те да бъдат
          публично достъпни.

\begin{lstlisting}[style=json,
                   caption=package\_locations.json по подразбиране,
                   label={lst:default-package-locations}]
{
    "package_locations": [
        "cpm-examples"
    ]
}
\end{lstlisting}
\end{itemize}
