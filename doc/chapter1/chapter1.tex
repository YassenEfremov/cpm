\graphicspath{ {./chapter1/images/} }

\chapter{Методи за управление на софтуерни пакети}


\section{Основни концепции}

\subsection{Видове системи за управление на пакети}

Съществуват основно два вида системи за управление на пакети:

\begin{itemize}
    \item Системни - управляват програмите на компютъра. Този вид системи са
          важен компонент от операционната система \acrshort{gnu}/Linux и
          нейните дистрибуции, като освен популярен софтуер управляват и важни
          системни библиотеки на самата операционната система (glibc,
          linux-util, и др.). Еднa от най-старите такива e dpkg за дистрибуцията
          Debian, на която са базирани десетки други дистрибуции. Други
          платформи също предоставят такива системи, като най-популярните за
          Windows и MacOS са съответно winget и homebrew. Те обаче не са
          интегрирани по същия начин както при \acrshort{gnu}/Linux и затова са
          по-слабо популярни сред повечето потребители.
    \item Езикови - управляват библиотеките на определен програмен език. Този
          вид системи са доста популярни сред модерните програмни езици. Те не
          зависят от операционната система. Главната им цел е лесното пакетиране
          и публикуване на библиотеки за конкретния език.
\end{itemize}

И двата вида системи решават съществуващите проблеми по подобни начини. Текущата
дипломна работа се фокусира върху втория вид системи, а именно тези, управляващи
библиотеките на определен програмен език.


\subsection{Видове софтуерни пакети}

При платформата \acrshort{gnu}/Linux софтуерните пакети се разпространяват чрез
вече интегрираната система за управление на пакети. Една програма или библиотека
обаче може да бъде пакетирана по два различни начина:

\begin{itemize}
    \item Двоичен пакет - този вид пакети са предварително компилирани. Те
          съдържат или изпълнимия двоичен код на програмата, или двоичният код
          на библиотеката (както и нейните header файлове, ако е написана на
          C/C++). Името им понякога завършва с наставката ``-bin''.
    \item Кодов пакет - този вид пакети се компилират при инсталация. Те
          съдържат самия код на програмата или библиотеката, като името им
          завършва с наставката ``-git''.
\end{itemize}


\subsection{Моделът клиент - регистър}

Всяка система за управление на пакети се състои от две основни части
\figref{fig:client-register}:

\begin{itemize}
    \item Клиент - (също наричан front-end) представлява програмата, с която
          потребителя взаимодейства и която инсталира, обновява и премахва
          посочените пакети. Тя може да предоставя графичен или конзолен
          потребителски интерфейс (сред системите за управление на библиотеки са
          по-популярни конзолните интерфейси).
    \item Регистър - (също наричан back-end) най-често представлява уеб
          приложение, което управлява базата данни, съхраняваща самите пакети и
          техните метаданни (метаданните са разгледани по-подробно в следващата
          точка).
\end{itemize}

Комуникацията между регистъра и клиентското приложение се осъществява
посредством приложно-програмен интерфейс (\acrshort{api}), предоставен от самия
регистър. Най-популярните решения за такъв интерфейс са \acrshort{rest}ful и
\acrshort{graphql} архитектурите.

\begin{figure}[h]
    \centering
    \includegraphics[width=\textwidth]{client-register}
    \caption{Моделът клиент - регистър}
    \label{fig:client-register}
\end{figure}


\subsection{Описание на пакетите (метаданни)}

Освен самия код на библиотеката, всеки пакет носи със себе си файл, описващ
различните му характеристики. Това са неговите така наричани метаданни.
Най-важните от тях са:

\begin{itemize}
    \item Име - всеки пакет има уникално име, което го идентифицира.
    \item Версия \cite{semver} - даден пакет може да има различни версии.
          Версията е под формата на три числа, разделени с точки (например
          2.11.3). Първото число обозначава главната версия, второто -
          второстепенната версия и третото - кръпката.
    \item Зависимости - даден пакет може да зависи от други пакети, които се
          наричат негови зависимости. Всеки пакет изброява своите зависимости и
          техните версии под формата на масив.
\end{itemize}

Често тези метаданни се съхраняват във файлови формати, които позволяват лесно
сериализиране, извличане и обработка на данните. Такива формати са например
\acrshort{json} и \acrshort{yaml}.


\subsection{Управление на зависимостите}

Едно от най-важните неща, за които се грижи системата за управление на пакети, е
управлението на зависимостите на даден пакет. Именно това ни позволява да
използваме вече съществуващи външни пакети в нашия проект, без да трябва да се
грижим за версиите на техните зависимости.

Зависимостите на даден пакет могат да бъдат представени като ориентиран граф.
Всеки връх в този граф представлява даден пакет, а дъгите между върховете ни
показват зависимостите им. Задачата на една системата за управление на
пакети е да разбере кои версии на кои пакети е нужно да бъдат инсталирани, така
че да няма повтарящи се, липсващи или конфликтни пакети. Това се постига чрез
алгоритъм за разрешаване на зависимостите.

\begin{figure}[h]
    \centering
    \includegraphics[width=0.75\textwidth]{dep-graph}
    \caption{Примерен граф на зависимостите}
\end{figure}

Можем да разделим зависимостите на два вида:

\begin{itemize}
    \item Директни - пакет \(A\) зависи пряко от пакет \(B\)
    \item Преходни - пакет \(A\) не зависи пряко от пакет \(B\). Вместо това
          пакет \(A\) зависи от пакет \(C\), който от своя страна зависи от
          пакет \(B\)
\end{itemize}

В даден момент графът на зависимостите може да стане толкова голям и сложен, че
самата операционна система да не може да го поддържа. Това се изразява в
прекалено дълги файлови пътища, получени в резултат на дълбоко вложени папки.
Начинът да се предотврати този проблем е така нареченото изравняване на
зависимостите (dependency flattening). При него всички пакети се съхраняват на
едно ниво в една папка, като зависимостите между тях се изразяват чрез
символични връзки (symlinks).

От метаданните на пакета се генерира така наречения lockfile, който описва
всички зависимости на текущия пакет и техните конкретни версии. Този файл също
се публикува заедно с кода на пакета и позволява точното пресъздаване на графа
на зависимостите му в различни среди.


\section{Съществуващи системи}

Най-популярните и разпространени системи за управление на пакети са тези за
езиците JavaScript и Python.

\begin{itemize}
    \item npm - системата на JavaScript (акроним за Node Package Manager). При
          нея всеки пакет се описва чрез файл на име package.json, в който са
          записани името на пакета, версията, лиценза, зависимостите и т.н.
    \item pip - системата на Python (рекурсивен акроним за PIP Installs
          Packages). За разлика от npm, при pip всеки пакет се описва с помощта
          на няколко файла - requirements.txt, setup.py и .toml.
\end{itemize}


\section{Съществуващи системи за C и C++}

Въпреки че няма общоприета система за управление на пакети за C и C++,
съществуват няколко не много популярни варианта, които си струва да бъдат взети
предвид.

\begin{itemize}
    \item Conan - система, предоставяща конзолен потребителски интерфейс и
          собствен регистър за теглене на пакети, наречен conan center, където
          са качени повечето известни библиотеки за C и C++. Системата зависи от
          езика Python и съответно може да бъде инсталирана чрез pip.
    \item vcpkg - система, разработена от Microsoft, написана на C++. vcpkg е
          интегриран в средите за разработка Visual Studio и Visual Studio Code.
          Предоставя конзолен потребителски интерфейс и регистър за теглене на
          пакети. Също позволява лесно пакетиране на вече съществуващи софтуерни
          библиотеки и интегрирането им с build системата CMake. За платформи
          различни от Windows инсталационния процес е малко нестандартен като
          изисква клониране на GitHub хранилището на проекта и изпълнението на
          няколко скрипта, които директно теглят от интернет изпълнимия файл на
          програмата.
\end{itemize}

За разлика от тези два съществуващи проекта, текущата дипломна работа се стреми
да разработи система за управление на пакети, която няма никакви външни
зависимости и е лесно достъпна както за Windows, така и за Linux.
