\chapter*{Увод}

В днешно време хората използват по-голямо количество софтуер от когато и да
било. Като започнем от ежедневно употребяваните програми, а именно уеб
браузърът, редакторите за документи, софтуерът за обработка на изображения,
видео и аудио, и стигнем чак до видеоигрите. Средностатистическият потребител,
е свикнал да си набавя всякакъв вид софтуер като просто потърси за него в
интернет пространството, намери достатъчно достоверно изглеждаща уеб страница,
от която да го свали, и изтегли приложението под формата на автоматизиран
инсталатор (файлове с разширение .msi, .deb, .dmg) или директно изпълним файл
(файлове с разширение .exe). Този подход обаче има недостатъци, най-големите от
които са удобството и сигурността. На всеки се е случвало при инсталирането на
даден софтуер да се изгуби из множеството уеб страници, опитващи се да те
заблудят като рекламират различни продукти или дори поставят фалшиви бутони за
инсталация, водещи към непознати подозрителни сайтове. Освен че по този начин си
губи времето, потребителят също така е изложен и на риск от заразяване със
зловреден софтуер. Един от начините да се избегне всичко това е използването на
система за управление на пакети (package manager). Тя позволява намирането,
инсталирането и актуализирането на софтуер безопасно под формата на пакети,
валидирани от истински хора и съхранявани в сигурни хранилища. Това има
множество предимства - унифицира се начина на работа със софтуерни пакети,
улеснява се работата на потребителя и се намалява риска от злонамерени действия
срещу него. Популярни примери за системи за управление на пакети са winget за
Windows, apt за Ubuntu Linux и homebrew за MacOS.

Системите за управление на пакети обаче могат да бъдат полезни и за самите
разработчици на софтуер. Една от основните им функции е да знаят от какви други
програми (библиотеки) е зависима дадена програма и съответно да управляват тези
зависимости. Големите софтуерни проекти са съставени от множество модули, които
зависят едни от други. Освен това всеки модул има различни версии, всяка от
които може да предоставя различни функционалности. Някои модули зависят от точно
определена версия на други модули. Така се създава една доста заплетена мрежа от
зависимости, която изисква постоянна поддръжка. Ако трябваше програмистът ръчно
да се грижи за всичко това, работата му щеше да се увеличи значително. За щастие
почти всеки език за програмиране в днешно време има своя система за управление
на пакети. Езикът Python има системата pip, езикът JavaScript има системата npm,
езикът Java има две такива системи - maven и gradle, и дори Rust, език за
системно програмиране, сравним със C и C++, има своя система, наречена cargo.
За жалост едни от езиците, които нямат общоприета система са именно C и C++. Има
опити да се направи такава, но въпреки това голяма част от програмистите
предпочитат да се грижат за зависимостите на проектите си ръчно. Една от
причините за това е факта, че и двата езика са създадени преди повече от 30
години и съответно следват по-различни практики от съвременните езици за
програмиране. Въпреки това C и C++ продължават да бъдат развивани и до ден
днешен, като и двата езика постепенно се сдобиват с нови по-модерни
функционалности.

Текущият проект цели реализирането на система за управление на пакети за
програмните езици C и C++. Самата система е написана на C и C++, като предоставя
конзолен потребителски интерфейс, посредством който могат да се инсталират и
премахват пакети под формата на хранилища, взети от онлайн платформата GitHub.
Системата също така позволява управление на зависимостите на даден пакет, както
и създаването на изцяло нови пакети, описани чрез файловия формат
\acrshort{json}.
